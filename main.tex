% Comment lines start with %
% LaTeX commands start with \
% This template was provided by Jennifer Welch for CSCE 222-200, Honors, Spring 2015

\documentclass[12pt]{article}  % This is an article with font size 12-point

% Packages add features
\usepackage{times}     % font choice
\usepackage{amsmath}   % American Mathematical Association math formatting
\usepackage{amsthm}    % nice formatting of theorems
\usepackage{amssymb}    % provides some symbols
\usepackage{latexsym}  % provides some more symbols
\usepackage{fullpage}  % uses most of the page (1-inch margins)

\setlength{\parskip}{.1in}  % increase the space between paragraphs

\renewcommand{\baselinestretch}{1.1}  % increase the space between lines

% Convenient renaming of symbols for logic formulas
\newcommand{\NOT}{\neg}
\newcommand{\AND}{\wedge}
\newcommand{\OR}{\vee}
\newcommand{\XOR}{\oplus}
\newcommand{\IMPLIES}{\rightarrow}
\newcommand{\IFF}{\leftrightarrow}
\newcommand{\powerset}[1]{\mathbb{P}(#1)}

% Actual content starts here.
\begin{document}

\begin{center}         % center all the material between begin and end
{\large                % use larger font
CSCE 222 (Carlisle), Discrete Structures for Computing \\  % \\ is line break
Fall 2020 \\
Homework 13}
\end{center}
\rule{6in}{.1pt}       % horizontal line 6 inches long and .1 point high
\begin{center}
{\large
Type your name below the pledge to sign\\
On my honor, as an Aggie, I have neither given nor received unauthorized aid on this academic work.\\
Sameer Hussain}
\end{center}

% blank line separates paragraphs.  First line of a paragraph is automatically
% indented.  

\rule{6in}{.1pt}       % horizontal line 6 inches long and .1 point high
                    
\noindent              % don't indent
{\bf Instructions:}    % \bf makes text boldface
                       % \em makes text emphasized (italics)

\begin{itemize}        % makes an itemized list
\item The exercises are from the textbook.  You are encouraged to work
      extra problems to aid in your learning; remember, the solutions to 
      the odd-numbered problems are in the back of the book.
\item Grading will be based on correctness, clarity, and whether your
      solution is of the appropriate length.
\item Always justify your answers.
\item Don't forget to acknowledge all sources of assistance in the section below, and write up your solutions on your own.
\item {\em Turn in .pdf file to Gradescope by the start of class on Tuesday, November 24, 2020.}  It is simpler to put each problem on its own page using the LaTeX clearpage command.
\end{itemize}


\rule{6in}{.1pt}       % horizontal line 6 inches long and .1 point high

{\bf Help Received:}    % \bf makes text boldface
\begin{itemize}
\item https://en.wikipedia.org/wiki/Wikipedia:LaTeX\_symbols
\item https://brilliant.org/wiki/finite-state-machines/
\item http://courses.ics.hawaii.edu/ReviewICS241/
\end{itemize}



\rule{6in}{.1pt}       % horizontal line 6 inches long and .1 point high

%---------------------------------------------------------------------

% \subsection makes a subsection heading; * leaves it unnumbered.
% (Usually subsections are inside sections, but the \section command
% used a font that was larger than I wanted.)
\subsection*{Exercises for Section 13.1:}     

\noindent
{\bf 4(a-c): (3 points).} \\
{\bf a)}\\
Start from start state $S$:\\
$1S$\\
$11S$\\
$111S$\\
$11100A$\\
$111000$\\
Therefore, $111000$ belongs to the language generated by $G$\\
{\bf b)}\\
$1S -> 11S -> 1100A$\\
Only one production step A $->$ 0. Strings that do not end in 0 cannot be in the language\\
{\bf c)}\\
Can only add 1s to string $S -> 1S$\\
$S-> 00A$\\
$= \{1^n 0^m | n \geq 0$ and $m \geq 3\}$


\noindent
{\bf 18(a,c): (2 points).} \\
{\bf a)}\\
$G = (V,T,S,P)$\\
$V = \{0,1,S,A\}$\\
$T = \{0,1\}$\\
Productions:
$S->0A$, $A->11A$, $A->\lambda$\\
{\bf c)}\\
$G = (V,T,S,P)$\\
$V = \{0,1,S,A\}$\\
$T = \{0,1\}$\\
Productions:
$S->0S0$, $S->A$, $S->\lambda$, $A->1A$, $A->\lambda$\\

\noindent
{\bf 24b: (1 point).} \\
$bbS$, $bbbcS$, $bbbcbbS$, $bbbcbba$\\

\clearpage

\subsection*{Exercises for Section 13.2:}     

\noindent
{\bf 4(a-c): (3 points).} \\
{\bf a)}\\
Input: 10001\\
Output: 00110\\
{\bf b)}\\
Output: 11110\\
{\bf c)}\\
Output: 10001\\


\noindent
{\bf 16: (2 points).} \\
We can consider three states:
In state 1, $s_0$, represents a position in the string that is divisble by 3.\\
State 2, $s_1$, represents a position $a$ in the string where $a mod 3 = 1$\\
State 3, $s_2$, represents a position $a$ in the string where $a mod 3 = 2$\\

\clearpage

\subsection*{Exercises for Section 13.3:}     

\noindent
{\bf 10(a,c,e): (3 points).} \\
{\bf a)}\\
$\{0,1\}^*$ contains any sequence of 0s and 1s so then it also contains the bit string 01101\\
{\bf c)}\\
$\{0\}^*$ contains any sequence of 010s.\\
$\{010\}^*\{0\}^*\{1\}$ contains a string with any sequence of zeros followed by a one\\
So it contains the bit string 01001\\
{\bf e)}\\
$\{0\}^*$ contains any sequence of 0s.\\
$\{00\}\{0\}^*\{01\}$ contains a string starting with two 0s followed by any sequence of 0s ending with 01\\
So it does not contain the bit string 00101 as it has multiple 1s\\


\noindent
{\bf 20: (2 points).} \\
$s_0$ is the initial state and $s_1$ and $s_3$ are the final states.\\
$s_0 -> s_1$ with any number of 1s followed by 1 or more number of 0s, after so $s_0 -> s_1 -> s_2 -> s_3$\\
Language is:\\
$\{1\}^*\{0\}\{0\}^*\cup\{1\}^*\{0\}^*\{1\}\{0,1\}\{0,1\}^*$\\


\noindent
{\bf 26: (2 points).} \\
\begin{center}
\begin{tabular}{ |c|c|c| } 
 \hline
 State & 0 & 1 \\ 
 $s_0$ & $s_0$ & $s_1$ \\ 
 $s_1$ & $s_2$ & $s_0$ \\ 
 $s_2$ & $s_3$ & $s_0$ \\ 
 $s_3$ & $s_3$ & $s_3$ \\ 
 \hline
\end{tabular}
\end{center}

\noindent
{\bf 38: (2 points).} \\
Proof by Contradiction\\
Assume that there exists a finite-state automation\\
Let $s_0$ be the start state and $s_1$, $s_2$ be two other states\\
$s_0$ needs to be a final state\\
Since string 0 does not contain even number of 0s and string 1 doesn't contain even numebr of 1s, need to be transitions from $s_0$ to other state for input\\
Let the input 0 be transition from $s_0 -> s_1$ and input 1 is from $s_0 -> s_2$\\
Strings 0 and 1 are not generated, then $s_1$ and $s_2$ are not final states.\\
$00$ and $11$ have to be accepted but $01$ not accepted. $011$ would not be accepted either.\\
If there is a transition from $s_1->s_1$ when input is 1, then string 000 would be accepted, but this is a contradiction as it does not contain an even number of 0s.\\
Therefore, by proof by contradiction, such a machine does not exist\\

\end{document}
